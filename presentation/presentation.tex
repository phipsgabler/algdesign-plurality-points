% works on TeX-Live 2015

\documentclass{beamer}
\usepackage{tgheros}
\usepackage[varqu, scaled]{inconsolata}
\usepackage{algpseudocode}
\usepackage{csquotes}
\usepackage{hyperref}
\usepackage{xcolor}
\usepackage{graphicx}
\usepackage{siunitx}
\usepackage{booktabs}
%\usepackage{tikz}
%\usepackage{tikz-uml}

% \lstset{
  % style=colored,
  % belowskip=0pt
  % basicstyle=\ttfamily\small\color{darkgray},
  % columns=[c]fixed,
  % gobble=4
% }

% colors
\providecolor{textgreen}{RGB}{59, 158, 72}
\providecolor{textblue}{RGB}{15, 100, 255}
\providecolor{textred}{RGB}{255, 51, 66}

\usetheme[compress]{Singapore}
\useinnertheme{circles}
\useoutertheme{split}
\setbeamercolor{structure}{fg=textblue}
\setbeamercolor{block body}{bg=normal text.bg!90!black}
\setbeamercolor{block title}{fg=black, bg=textblue!90}

% smaller footnotes; see: http://tex.stackexchange.com/a/192652/46356
\setbeamertemplate{footnote}{%
  \tiny%
  \parindent 1em\noindent%
  \raggedright
  \hbox to 1.8em{\hfil\insertfootnotemark}\insertfootnotetext\par%
}%
\setlength\footnotesep{0pt}

\author{Philipp Gabler}
\title[Plurality Points\\ and Condorcet Points\\ in Euclidean Space]{%
  Computing Plurality Points\\ and Condorcet Points\\ in Euclidean Space}
\date{}


% math stuff
\usepackage{amsmath}

\newcommand{\RR}{\mathbb{R}}
\newcommand{\dotleq}{\mathrel{\dot{\leq}}}
\newcommand{\dotgeq}{\mathrel{\dot{\geq}}}
\newcommand{\preferers}[3][\succ]{[#2 #1 #3]}
\newcommand{\npreferers}[3][\succ]{\lvert\preferers[#1]{#2}{#3}\rvert}
\newcommand{\order}[1]{\ensuremath{\mathcal{O}(#1)}}

\newcommand{\eqspacing}[2]{%
  \setlength\abovedisplayskip{#1}%
  \setlength\belowdisplayskip{#2}%
}


% biblatex
\usepackage[backend=biber]{biblatex}
% patch doi entries to favour openly available links
\DeclareFieldFormat{doi}{\mkbibacro{DOI}\addcolon\space\href{http://doai.io/#1}{#1}}

\addbibresource{refs.bib}

\def\qedsymbol{}

%%%%%%%%%%%%%%%%%%%%%%%%%%%%%%%%%%%%%%%%%%%%%%%%%%%%%%%%%%%%%%%%%%%%%%%%%%%%%%%%%%%%%%%%%%%%%%%%%%%%
\begin{document}
\beamertemplatenavigationsymbolsempty
% \addtobeamertemplate{block begin}{%
%   \setlength\abovedisplayskip{0pt}%
%   \setlength\belowdisplayskip{0pt}}
\addtobeamertemplate{proof end}{\vspace{-1em}} %remove spacing for \qed

\begin{frame}
  \maketitle
\end{frame}

\section{Introduction}

\begin{frame}{Outline}
  \begin{enumerate}
  \item Informal introduction
  \item Notation
  \item Idea of algorithm, neccessary lemmata by case
  \item Algorithm
  \item Analysis of algorithm
  \end{enumerate}
\end{frame}

\begin{frame}{Informal description}
  We can represent opinion standpoints by points in space.  There are several ways define a
  (spacial) equilibrium between these quantified \enquote{opinions}.
  
  \begin{itemize}
  \item Voters: Multiset in Euclidean space (\(\RR^d\) with \(\ell_2\) norm)
  \item Plurality point: Closer to at least as many voters as any other point
  \item Condorcet point: No other point is closer to an absolute majority of voters
  \end{itemize}

  Problem: given the voters, find plurality and Condorcet points.
\end{frame}


%%%%%%%%%%%%%%%%%%%%%%%%%%%%%%%%%%%%%%%%%%%%%%%%%%%%%%%%%%%%%%%%%%%%%%%%%%%%%%%%%%%%%%%%%%%%%%%%%%%%
\section{Preliminaries}

\subsection{Notation \& Definitions}
\begin{frame}{Notation}
  \begin{definition}[Voters]
    Voters: \(V = \{v_1, \dotsc, v_n\} \subseteq \RR^d \) (a multiset, in general). The components
    are denoted as \(v_i = (v^{(1)}_i, \dotsc, v^{(d)}_i)\).
  \end{definition}
  % \begin{definition}[Dimensionwise comparison]
  %   \(v \dotleq w \Leftrightarrow \forall j: v^{(j)} \leq w^{(j)}\), and accordingly for other
  %   relations.
  % \end{definition}
  \begin{definition}[Multidimensional medians]
    \eqspacing{0pt}{0pt}
    Informally: at least half the voters are on each side, per dimension:
    \begin{align*}
      \mathcal{M}_V = \{ \theta :\; &\lvert\{v \in V : \forall j: v^{(j)} \leq \theta^{(j)}\}\rvert \geq n/2 \\
                           \wedge\; &\lvert\{v \in V : \forall j: v^{(j)} \geq \theta^{(j)}\}\rvert \geq n/2 \}
    \end{align*}
  \end{definition}
\end{frame}


\begin{frame}{Notation}
  \begin{definition}[Median corners]
    \eqspacing{0pt}{0pt}
    Let \(V^{(i)}\) be the \(i\)-th coordinate of points in \(V\), and \(x^{(i)}_l\) and
    \(x^{(i)}_h\) denote the \(\lceil n/2 \rceil\)-th and \(\lceil (n+1)/2 \rceil\)-th smallest
    numbers in \(V^{(i)}\). Then the set
    \begin{equation*}
      C = \{(x^{(i)}_{t_1}, \dotsc, x^{(i)}_{t_d}) : t_1, \dotsc, t_d \in \{l, h\}\}
    \end{equation*}
    collects all corners of \(\mathcal{M}_V\). Clearly, \(\lvert C\rvert \leq 2^d\). 
  \end{definition}
  \begin{definition}[Half spaces]
    \eqspacing{1ex}{0pt}
    Given a line \(L\), we can partition \(\RR^d = L^+ \cup L \cup L^-\), where \(L^+\) and \(L^-\)
    are the open half spaces separated by \(L\). We define:
    \begin{gather*}
      V_L = L \cap V, \quad V^+_L = L^+ \cap V, \quad V^-_L = L^- \cap V \\
      n_L = \lvert V_L \rvert, \quad n^+_L = \lvert V^+_L \rvert, \quad n^-_L = \lvert V^-_L \rvert
    \end{gather*}
  \end{definition}
\end{frame}


\begin{frame}{Formal definition}
  \begin{definition}[Preference]
    \begin{itemize}
    \item \(v_i\) prefers \(\theta_1\) to \(\theta_2\) if \(d(v_i, \theta_1) < d(v_i, \theta_2)\)
    \item \(\preferers{\theta_1}{\theta_2}  := \{v \in V : d(v, \theta_1) < d(v, \theta_2)\}\), and
      accordingly for other relations
    \end{itemize}
  \end{definition}
  \begin{definition}[Plurality point (PP)]
    \eqspacing{0pt}{0pt}
    \(\Delta \in \RR^d\) is a plurality point iff
    \begin{equation*}
      \forall \theta \in \RR^d : \npreferers{\theta}{\Delta} \leq \npreferers{\Delta}{\theta}
    \end{equation*}
  \end{definition}
  \begin{definition}[Condorcet point (CP)]
    \eqspacing{0pt}{0pt}
    \(\Delta \in \RR^d\) is a Condorcet point iff
    \begin{equation*}
      \forall \theta \in \RR^d : \npreferers{\theta}{\Delta} \leq n/2
    \end{equation*}
  \end{definition}
\end{frame}


\begin{frame}{Equivalence of PPs and CPs}
  \begin{lemma}[Equivalence]  % Lemma 1
    In \(\RR^d\), a point is a plurality point iff it is a Condorcet point.
  \end{lemma}
  \begin{proof}
    \begin{enumerate}
    \item A PP is a CP by definition.
    \item
      \begin{itemize}
      \item Assume \(\Delta\) is a CP, but not a PP, so
        \(\exists \delta: \npreferers{\Delta}{\delta} < \npreferers{\delta}{\Delta}\).
      \item Let \(m\) be the midpoint of \(\Delta, \delta\).  Then for each
        \(v \in \preferers[\succeq]{\delta}{\Delta}\),
        \(d(v, m) < d(v, \Delta)\).
      \item Thus, \(\npreferers{m}{\Delta} > n/2\), which is a contradiction.
      \item Therefore, a CP is also a PP.
      \end{itemize}
    \end{enumerate}
  \end{proof}

  From now on, we constrain ourselves to PPs in \(\RR^2\).  
\end{frame}


\begin{frame}{Idea behind algorithm}
  The presented algorithm works by successively cutting down the space of possible PPs via a series
  of case distinctions, until only \order{1} candidate points remain, which can be checked
  exhaustively. For this purpose, the following series of lemmata is employed. \\~\\

  Furthermore:
  \begin{definition}[]
    \(\Delta_V\) denotes the set of all PPs of \(V\)
  \end{definition}
\end{frame}


\subsection{Supporting Lemmata}

\begin{frame}{Collinear case} 
  \begin{lemma}[Collinear points] % Lemma 2
    If all voters in \(V\) are collinear, then
    \begin{enumerate}
    \item \(\Delta_V = \mathcal{M}_V\), and
    \item \(\lvert\Delta_{v}\rvert \geq 1\).
    \end{enumerate}
  \end{lemma}
  \begin{proof}
    \begin{enumerate}
    \item has been shown by Hansen \& Thisse (1981) for CCs in \(\RR^2\). By the
      equivalence lemma, this holds also for PPs.
    \item By definition, \(\mathcal{M}_V \neq \emptyset\). When \(\lvert V\rvert\) is even, we may
      have \(\lvert\mathcal{M}_V\rvert = \lvert\Delta_V\rvert > 1\).
    \end{enumerate}
\end{proof}
\end{frame}


\begin{frame}{\enquote{Tukey depth condition}}
  This neccessary and sufficient condition is used to decide whether a candidate point is a valid
  PP.
  \begin{lemma}[]  % Lemma 3
    In \(\RR^2\), \(\Delta\) is a plurality point iff for any line \(L\) through \(\Delta\), \(n^+_L
    \leq n/2\) and \(n^-_L \leq n/2\). 
  \end{lemma}
%   \begin{proof}
%   \end{proof}
  \eqspacing{1ex}{1ex}
  It is equivalent checking if
  \begin{equation*}
    \min_{L \in \mathcal{L}_{\Delta}} \{\lvert V \cup \gamma \rvert : \text{\(\gamma\) is a closed halfspace
      separated by \(L\)}\} \geq n/2,
  \end{equation*}
  where \(\mathcal{L}_{\Delta}\) is the space of lines through \(\Delta\). The quantity on the left
  is called \textit{Tukey depth} of \(\Delta\) with respect to \(V\).
\end{frame}


\begin{frame}{Non-collinear case \& medians}
  \begin{lemma}[Non-collinear candidates]  % Lemma 4
    If not all voters in \(V\) are collinear, then
    \begin{enumerate}
    \item \(\Delta_V \subseteq \mathcal{M}_V\), and
    \item \(\lvert\Delta_{v}\rvert = 0\), or \(\lvert\Delta_{v}\rvert = 1\).
    \end{enumerate}
  \end{lemma} ~\\
%   \begin{proof}
%   \end{proof}
  So, a plurality point must be a median, but not vice versa.  We can use this to cut down the
  searched space to the median; however, this is still infinite.
\end{frame}


\begin{frame}{\(\Delta\) is in \(\mathcal{M}_V\)}
  These lemmata reduce the possibilities in the non-collinear case further to \order{1} candidate
  points.
  \begin{lemma}[\(\Delta \in V\)]  % Lemma 5
    If \(\Delta\) is a PP of \(V\), and \(\Delta \in V\), then \(\Delta \in C\), with
    \(V \cup C \leq 2^d\).
  \end{lemma}
  \begin{lemma}[\(\Delta \not\in V\)] % Lemma 6
    If \(\Delta\) is a PP of \(V\), \(\Delta \not\in V\), and \(L\) is a separating line through
    \(\Delta\) (ie., \(n_L = 0\)), then \(\Delta\) is the intersection of the internal tangents of
    the convex hulls of \(V_L^+\) and \(V_L^-\).
  \end{lemma}
\end{frame}


%%%%%%%%%%%%%%%%%%%%%%%%%%%%%%%%%%%%%%%%%%%%%%%%%%%%%%%%%%%%%%%%%%%%%%%%%%%%%%%%%%%%%%%%%%%%%%%%%%%%
\section{Algorithm}

\subsection{Description}
\let\oldcomment\Comment
\renewcommand{\Comment}[1]{\oldcomment{\textit{#1}}}

\begin{frame}{Subroutines}
  \begin{itemize}
  \item \textsc{Select}(\(S, n\)) returns the n-th smallest point in S. This can be calculated in
    \order{|S|} time by the median-of-medians selection algorithm from Cormen et. al. Used for
    median calculation.
  \item \textsc{VerifyCandidates}(\(C, V\)) returns all points from C whose Tukey depth in is
    greater or equal \(|V|/2\) (ie., satisfying the condition in the \enquote{Tukey depth lemma}
    above). Tukey depth of one point can be computed in \order{|V| \log |V|} time with an algorithm
    by Rousseeuw and Struyf (1998).
  \end{itemize}
\end{frame}


\begin{frame}{Subroutines}
  \begin{itemize}
  \item \textsc{InternalTangentIntersection}(\(V, W\)) returns the intersection of the internal
    tangents between the convex hulls of V and W. The tangents can be found by solving the linear
    program
    \begin{align*}
      &\max_{k, d}\, F(k, d) = k, \quad \text{s.t.}\\
      &\forall\; v \in V: v_y \leq k  v_x + d \quad \text{and} \\
      &\forall\; w \in W: w_y \geq k  w_x + d,
    \end{align*}
    and the opposite formulation (minimizing with \(V\) and \(W\) swapped). Linear programming with
    fixed dimension can be done in linear time in the number of constraints, by Megiddo (1984).
  \end{itemize}
\end{frame}


\algrenewcommand\alglinenumber[1]{\tiny #1:}
\begin{frame}{Pseudocode}
  \tiny
  \begin{columns}[t]
    \column{0.45\textwidth}
    \begin{algorithmic}[1]
      \Procedure{PluralityPoint2D}{$V$}
      \If{all voters are collinear}
        \State return medians of $V$
      \Else
        \Comment{Corners of medians}
        \State $x_h$ = \Call{Select}{$V_x$, $\lceil (n + 1) / 2 \rceil$}
        \State $x_l$ = \Call{Select}{$V_x$, $\lceil n / 2 \rceil$}
        \State $y_h$ = \Call{Select}{$V_y$, $\lceil (n + 1) / 2 \rceil$}
        \State $y_l$ = \Call{Select}{$V_y$, $\lceil n / 2 \rceil$}
        \State $C$ = $\{(x_h, y_h), (x_h, y_l), (x_l, y_h), (x_l, y_l)\}$
        \Statex
        \If{$|C| = 1$}
          \Comment{$n$ odd, or $C$ degenerate}
          \State return \Call{VerifyCandidates}{$C, V$}
        \Else
          \State $P$ = \Call{VerifyCandidates}{$V \cap C, V$}
          \If{$P \not= \emptyset$}
            \Comment{$\Delta \in V$}
            \State return $P$
          \algstore{pp}
    \end{algorithmic}
    \column{0.55\textwidth}
    \begin{algorithmic}[1]
          \algrestore{pp}
          \Else
            \Comment{$\Delta \not\in V$}
            \If{$x_h = x_l$}
              \State $V_a$ = $\{v \in V : v^{(2)} \leq y_h\}$
              \State $V_b$ = $\{v \in V : v^{(2)} \geq y_h\}$
            \Else
              \State $V_a$ = $\{v \in V : v^{(1)} \leq x_h\}$
              \State $V_b$ = $\{v \in V : v^{(1)} \geq x_h\}$
            \EndIf
            \Statex  
            \State $p$ = \Call{InternalTangentIntersection}{$V_a, V_b$}
            \State return \Call{VerifyCandidates}{$\{p\}, V$}
          \EndIf
          \EndIf
        \EndIf  
      \EndProcedure
    \end{algorithmic}
  \end{columns}
\end{frame}


\subsection{Analysis}
\begin{frame}{Analysis}
  \begin{itemize}
  \item Detecting collinearity (line 2): \order{n}
  \item Selecting median corners (lines 5--8): \order{n}
  \item Verifying candidates by calculating Tukey depth (lines 11, 13, 25): \order{n \log n}
  \item Computing internal tangent intersection (line 24): \order{n}
  \item Therefore: \order{n \log n} overall complexity
  \end{itemize}
\end{frame}

%%%%%%%%%%%%%%%%%%%%%%%%%%%%%%%%%%%%%%%%%%%%%%%%%%%%%%%%%%%%%%%%%%%%%%%%%%%%%%%%%%%%%%%%%%%%%%%%%%%%
\section*{}
\begin{frame}{Conclusion}
  \begin{itemize}
  \item In Euclidean space, plurality points and Condorcet points are equivalent notions
  \item In \(\RR^2\), they can be calculated in \(\mathcal{O}(n \log n)\) time, using the presented
    algorithm
  \item This can be generalized to \(\RR^d\) in \(\mathcal{O}(n^{d-1} \log n)\) time
  \end{itemize}

  \vspace{1cm}
  \begin{block}{Working implementation (in Julia)}
    \url{https://github.com/phipsgabler/plurality-points}
  \end{block}
\end{frame}


\begin{frame}{Bibliography}
  \renewcommand*{\bibfont}{\small}
  \nocite{*}
  \printbibliography
\end{frame}

\end{document}